%TODO: Presentation 10 (Software Testing)

\documentclass[c, aspectratio=169]{beamer}
\usetheme[nofirafonts]{focus}
%\usepackage{beamercolorthememagpie}
\usepackage{animate}
\usepackage{anyfontsize}
\usepackage{comment}
\usepackage{multicol}
\usepackage{listings}
\usepackage{tgadventor}
\usepackage{verbatim}
\usepackage{xcolor}

\title{CPE 212 Review Guide}
\author{Joshua Bays}\date{Fall 2024}
\institute{Univeristy of Alabama in Huntsville}

\definecolor{main}{RGB}{00, 66, 142}
\definecolor{background}{RGB}{240, 247, 255}
%\definecolor{background}{RGB}{0, 0, 0}

\hypersetup{colorlinks=true, linkcolor=black, urlcolor=blue}
\footlineinfo{CPE 212 Review}

\definecolor{clr-background}{RGB}{255,255,255}
\definecolor{clr-text}{RGB}{0,0,0}
\definecolor{clr-string}{RGB}{163,21,21}
\definecolor{clr-namespace}{RGB}{0,0,0}
\definecolor{clr-preprocessor}{RGB}{128,128,128}
\definecolor{clr-keyword}{RGB}{0,0,255}
\definecolor{clr-type}{RGB}{43,145,175}
\definecolor{clr-variable}{RGB}{0,0,0}
\definecolor{clr-constant}{RGB}{111,0,138} % macro color
\definecolor{clr-comment}{RGB}{0,128,0}

\lstdefinestyle{mystyle}{
	language=C++,
	backgroundcolor=\color{clr-background},
	basicstyle=\color{clr-text}, % any text
	stringstyle=\color{clr-string},
	identifierstyle=\color{clr-variable}, % just about anything that isn't a directive, comment, string or known type
	commentstyle=\color{clr-comment},
	directivestyle=\color{clr-preprocessor}, % preprocessor commands
	% listings doesn't differentiate between types and keywords (e.g. int vs return)
	% use the user types color
	keywordstyle=\color{clr-type},
	keywordstyle={[2]\color{clr-constant}}, % you'll need to define these or use a custom language
	tabsize=4,
}
\lstset{style=mystyle}
\lstset{directives={\#include}}


\begin{document}
\begin{frame}
\titlepage
\end{frame}

\section*{About this Document}
\begin{frame}\frametitle{About this Document}
\begin{itemize}
\item This document is intended to serve as a useful reference and study tool for the CPE 212 final exam.
\item The code snippets presented in this document are designed to be basic representations of different concepts, and may differ from the methods presented in lectures
\item Any suggestions for additions or changes can be made to Joshua Bays by emailing \href{mailto:jb0401@uah.edu}{jb0401@uah.edu} or by sending a message via Canvas
\end{itemize}
\end{frame}

\begin{frame}\frametitle{Table of Contents}
\begin{multicols}{2}
\tableofcontents
\end{multicols}
\end{frame}

\section{Classes}
\begin{frame}\frametitle{Classes - Overview}
\begin{itemize}
\item Class: Custom model of abstract data type (ADT)
\item Object: Instance of a class
\item Member types
	\begin{itemize}
	\item Public: Can be accessed directly from outside of the class
	\item Private: Can be accessed directly only from within the class
	\item Protected: Can be inherited within derived classes
	\end{itemize}
\item Member functions
	\begin{itemize}
	\item Constructor: Initialize an object
	\item Transformers: Change an object's state
	\item Observers: Get (but not change) an object's state
	\item Iterators: Process all components within an ADT
	\item Destructor: Properly clean up and object (Ex: de-allocate memory)
	\end{itemize}
\end{itemize}
\end{frame}

\begin{frame}\frametitle{Classes - Inheritance}
\begin{itemize}
\item Inheritance: Reuse existing class code for another class
\item Multiple inheritance: Inheriting code from multiple classes
\item Parent/Base class: The class being inherited from
\item Derived class: A class that inherits from another class
\item Virtual function: A member function that can be redefined by an inherited class
\item Friend function: Non-member function that can access private members
\end{itemize}
\end{frame}

\begin{frame}\frametitle{Classes - Code}
Inheritance
\lstinputlisting[basicstyle=\tiny, firstline=3, lastline=20]{code/inheritance.cpp}
\lstinputlisting[basicstyle=\tiny, firstline=30, lastline=36]{code/inheritance.cpp}
\end{frame}

\begin{frame}[fragile]\frametitle{Classes - Code}
Inheritance (cont.)
\begin{center}
\lstinputlisting[basicstyle=\tiny, firstline=22, lastline=28]{code/inheritance.cpp}
\end{center}

Output
\verbatiminput{code/inheritance-output.txt}
\end{frame}

\section{Pointers}
\begin{frame}\frametitle{Pointers - Overview}
\begin{itemize}
\item Pointer: Variable that stores the memory address of another variable
\item Dereferencing: Access the value stored in the location stored in the pointer
\item Static allocation: Memory allocated at compile time
\item Dynamic allocation: Memory allocated during program runtime
\item Heap: Free memory for dynamic allocation
\end{itemize}
\end{frame}

\begin{frame}\frametitle{Pointers - Overview}
\begin{itemize}
\item Memory leak: Memory dynamically allocated but not deallocated
\item Garbage: Locations in memory that can not be accessed any more
\item Inaccessible object: Dynamically allocated variable without a pointer
\item Dangling pointer: A pointer that points to memory that has been deallocated
\end{itemize}
\end{frame}

\begin{frame}\frametitle{Pointers - Dynamic Allocation}
\begin{itemize}
\item Use the new keyword to allocate memory in C++
\item Use the delete keyword on a pointer to deallocate memory in C++
\end{itemize}
\end{frame}

\begin{frame}\frametitle{Pointers - Code}
Static allocation
\lstinputlisting[basicstyle=\tiny, firstline=3]{code/pointer1.cpp}

Output
\verbatiminput{code/pointer1-output.txt}
\end{frame}

\begin{frame}\frametitle{Pointers - Code}
Dynamic Allocation
\lstinputlisting[basicstyle=\tiny, firstline=4]{code/pointer2.cpp}

Output
\verbatiminput{code/pointer2-output.txt}
\end{frame}

\section{Exception Handling}
\begin{frame}\frametitle{Exception Handling - Overview}
\begin{itemize}
\item Robustness: How well a program can recover from an error
\item Error types
	\begin{itemize}
	\item Unexpected user input
	\item Hardware issues
	\item Software issues
	\end{itemize}
\item Ways to handle errors
	\begin{itemize}
	\item Print an error message
	\item Return an unusual value (Ex: -1)
	\item Use a status variable as an error flag
	\item Use assertions to prevent further code execution
	\item Exception handling
	\end{itemize}
\end{itemize}
\end{frame}

\begin{frame}\frametitle{Exception Handling - Overview}
\begin{itemize}
\item Exception: Unexpected event that requires special processing
\item Exception handler: Code designed to address a specific exception
\end{itemize}
\end{frame}

\begin{frame}\frametitle{Exception Handling - Try/Throw/Catch}
\begin{itemize}
\item Try: Execute code that may cause and exception within its own block
\item Throw: If an error is detected terminate the program or execute code to address the exception by ``throwing'' an error
\item Catch: Address the exception based on the type of error provided by the throw statement
\end{itemize}
\end{frame}

\begin{frame}\frametitle{Exception Handling - Code}
Basic exception handling
\lstinputlisting[basicstyle=\tiny, firstline=3]{code/try-throw-catch.cpp}
\end{frame}

\section{Software Engineering}
\begin{frame}\frametitle{Software Engineering - Overview}
\begin{itemize}
\item Attributes of good software
	\begin{itemize}
	\item Works
	\item Can be easily modified
	\item Is reusable
	\item Is completed within time and budget requirements
	\end{itemize}
\item Software engineering: The proper application of the principles of design, production, and maintenance of software
	\begin{itemize}
	\item Technical challenges
	\item Project management
	\end{itemize}
\item Defects in code
	\begin{itemize}
	\item About 1 error is created for every 10 lines of code
	\item 75\% of a code's cost is in maintenance of that code
	\end{itemize}
\item Software process: The process by which software is developed and maintained
\end{itemize}
\end{frame}

\begin{frame}\frametitle{Software Engineering - Parts of Software Development}
\begin{itemize}
\item Requirements: High-level description of the product
\item Specification: Detailed description containing functional requirements and constraints
\item Design: Architectural (high-level) and detailed (low-level) design of the product
\item Implementation: Converting the design into code
\item Testing/Verification: Finding and fixing errors and demonstrating that the product works correctly
\item Postdelivery/Maintenance: Correct errors found by users and enhancing functionality
\end{itemize}
\end{frame}

\begin{frame}\frametitle{Software Engineering - Processes}
\begin{itemize}
\item Waterfall process: Each step of the process is an input for the next step
\item Agile process: Emphasizing individuals/interactions and working software over specific processes in order to enable quick changing and customer collaboration
\item Scrum: Work is designed to be done in short periods called ``sprints,'' with daily work being determined by the needs of the current sprint
\end{itemize}
\end{frame}

\begin{frame}\frametitle{Software Engineering - Testing Overview}
\begin{itemize}
\item Testing: Trying to discover errors within a program
\item Debugging: Removing known errors from a program
\item Driver: A program specifically designed to test a part of code
\item Stub: Dummy code designed to simulate real-life use cases
\item Assertion: A statement that is either true or false
\item Precondition: An assertion that must be true in order for a postcondition to be returned
\item Postcondition: An assertion that is expected from a certain precondition
\end{itemize}
\end{frame}

\begin{frame}\frametitle{Software Engineering - Testing Hierarchy and Types}
\begin{itemize}
\item Deskchecking: Informal checking by the developer
\item Unit testing: Formally testing individual parts of a program by themselves
\item Integration testing: Formally and systematically testing a part of a program within the larger code base
\item Acceptance testing: Testing the program with real data in its real environment
\item Regression testing: Testing a program following modifications
\item Black-box testing: Testing a program by its inputs and outputs
\item Clear-box testing: Testing a program utilizing knowledge of its structure
\end{itemize}
\end{frame}

\begin{frame}\frametitle{Software Engineering - Verification VS Validation}
\begin{itemize}
\item Verification: The program works properly
\item Validation: The program satisfies the needs of the problem
\end{itemize}
\end{frame}

\section{Testing}

\section{Stacks}
\begin{frame}\frametitle{Stacks - Overview}
\begin{itemize}
\item Stack: Specially organized list
	\begin{itemize}
	\item LIFO structure (Last In, First Out)
	\item Data entry is only through the top of the stack
	\end{itemize}
\item Basic stack operations:
\begin{itemize}
\item Push: Add an item from the top of the stack
\item Pop: Remove the top item from the stack
\item Top: Observe the top item from the stack
\item IsEmpty: Returns if the stack has no elements on it
\item IsFull: Returns if the stack is at its maximum capacity
\item MakeEmpty: Remove all elements from the stack
\end{itemize}
\item Different methods exist to implement stacks
	\begin{itemize}
	\item Array-based: Less memory used, but harder to resize
	\item Linked node-based: Easier to resize, but more memory used
	\end{itemize}
\end{itemize}
\end{frame}

\begin{frame}\frametitle{Stacks - Code}
Basic stack code (Linked list)
\lstinputlisting[basicstyle=\tiny, firstline=4, lastline=30]{code/stack.cpp}
%\lstinputlisting[basicstyle=\tiny, firstline=12, lastline=27]{code/stack.cpp}
\end{frame}

\begin{frame}\frametitle{Stacks - Code}
Basic stack code (cont.)
\lstinputlisting[basicstyle=\tiny, firstline=47, lastline=55]{code/stack.cpp}
\end{frame}

\begin{frame}\frametitle{Stacks - Code}
Basic stack code (cont.)
\lstinputlisting[basicstyle=\tiny, firstline=57, lastline=75]{code/stack.cpp}
\end{frame}

\begin{frame}\frametitle{Stacks - Code}
Basic stack code (cont.)
\lstinputlisting[basicstyle=\tiny, firstline=77, lastline=99]{code/stack.cpp}
\end{frame}

\begin{frame}\frametitle{Stacks - Code}
Basic stack code (cont.)
\lstinputlisting[basicstyle=\tiny, firstline=101]{code/stack.cpp}
\end{frame}

\section{Lists}
\begin{frame}\frametitle{Lists - Overview}
\begin{itemize}
\item List: Linear collection of homogeneous items
	\begin{itemize}
	\item Can be sorted or unsorted
	\end{itemize}
\item Basic list operations:
	\begin{itemize}
	\item IsEmpty: 
	\item IsFull:
	\item Length: 
	\item Insert: 
	\item Delete: 
	\item IsPresent
	\end{itemize}
\end{itemize}
\end{frame}

\begin{frame}\end{frame}
\begin{frame}\end{frame}

\section{Queues}
\begin{frame}\end{frame}
\begin{frame}\end{frame}
\begin{frame}\end{frame}

\section{Generic Programming}
\begin{frame}\end{frame}
\begin{frame}\end{frame}
\begin{frame}\end{frame}

\section{Recursion}
\begin{frame}\end{frame}
\begin{frame}\end{frame}
\begin{frame}\end{frame}

\section{Trees}
\begin{frame}\end{frame}
\begin{frame}\end{frame}
\begin{frame}\end{frame}

\section{Heaps}
\begin{frame}\end{frame}
\begin{frame}\end{frame}
\begin{frame}\end{frame}

\section{Graphs}
\begin{frame}\end{frame}
\begin{frame}\end{frame}
\begin{frame}\end{frame}

\section{Searching}
\begin{frame}\frametitle{Searching - Overview}
\begin{itemize}
\item Searching: Finding a specific item from a set of data
	\begin{itemize}
	\item Efficiency: Program performance is improved
	\item Data retrieval: Specific data is quickly found in a large dataset
	\item Problem solving: Data needs to be found in order to solve problems
	\end{itemize}
\item Different methods of searching
	\begin{itemize}
	\item Linear search
	\item Binary search
	\item Hashing
	\end{itemize}
\end{itemize}
\end{frame}

\begin{frame}\frametitle{Searching - Linear Search}
\begin{itemize}
\item Linear search: Sequentially search through a set of data until the value is found
\item Complexity:
	\begin{itemize}
	\item Best case: $O(1)$ (The first element)
	\item Worst case: $O(n)$ (The last element)
	\end{itemize}
\item Use cases:
	\begin{itemize}
	\item Small dataset
	\item Unordered datasets
	\item Linked lists
	\end{itemize}
\end{itemize}
\end{frame}

\begin{frame}\frametitle{Searching - Code}
Basic linear search code
\lstinputlisting[basicstyle=\tiny, firstline=3]{code/linear-search.cpp}

Output
\verbatiminput{code/linear-search-output.txt}
\end{frame}

\begin{frame}\frametitle{Searching - Binary Search}
\begin{itemize}
\item Binary search: Continually divide the search area in half, comparing the middle value to the target value
\item Complexity
	\begin{itemize}
	\item Best case: $O(1)$ (The first element)
	\item Worst case: $O(\log n)$ (The last element)
	\end{itemize}
\item Use cases:
	\begin{itemize}
	\item Data must be sorted
	\item Random access should be a constant time function
	\end{itemize}
\end{itemize}
\end{frame}

\begin{frame}\frametitle{Searching - Code}
Basic binary search code
\lstinputlisting[basicstyle=\tiny, firstline=3]{code/binary-search.cpp}

Output
\verbatiminput{code/binary-search-output.txt}
\end{frame}

\begin{frame}\frametitle{Searching - Hashing}
\begin{itemize}
\item Hashing: Data storage technique designed to allow $O(1)$ search time
	\begin{itemize}
	\item Assign key-value pairs through a function to data inputs
	\item Hash function used to store the element and to find if the element exists in the dataset
	\item Tradeoff of memory space for access speed
	\end{itemize}
\item Collision: Repeated outputs for different inputs
	\begin{itemize}
	\item Must be addressed with hash function (Ex: Offsetting the value)
	\end{itemize}
\end{itemize}
\end{frame}

\begin{frame}\frametitle{Searching - Code}
Basic hashing code
\lstinputlisting[basicstyle=\tiny, firstline=3, lastline=25]{code/hashing.cpp}
\end{frame}

\begin{frame}\frametitle{Searching - Code}
Basic hashing code (Cont.)
\lstinputlisting[basicstyle=\tiny, firstline=27]{code/hashing.cpp}
Output
\verbatiminput{code/hashing-output.txt}
\end{frame}

\section{Sorting}
\begin{frame}\end{frame}
\begin{frame}\end{frame}
\begin{frame}\end{frame}

\end{document}
