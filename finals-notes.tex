% TODO
%	Lists
%	Queues
%	Trees

\documentclass[c, aspectratio=169]{beamer}
\usetheme[nofirafonts]{focus}
%\usepackage{beamercolorthememagpie}
\usepackage{animate}
\usepackage{anyfontsize}
\usepackage{comment}
\usepackage{multicol}
\usepackage{listings}
\usepackage{tgadventor}
\usepackage{verbatim}
\usepackage{xcolor}

\title{CPE 212 Review Guide}
\author{Joshua Bays}\date{Fall 2024}
\institute{Univeristy of Alabama in Huntsville}

\definecolor{main}{RGB}{00, 66, 142}
\definecolor{background}{RGB}{240, 247, 255}
%\definecolor{background}{RGB}{0, 0, 0}

\hypersetup{colorlinks=true, linkcolor=black, urlcolor=blue}
\footlineinfo{CPE 212 Review}

\definecolor{clr-background}{RGB}{255,255,255}
\definecolor{clr-text}{RGB}{0,0,0}
\definecolor{clr-string}{RGB}{163,21,21}
\definecolor{clr-namespace}{RGB}{0,0,0}
\definecolor{clr-preprocessor}{RGB}{128,128,128}
\definecolor{clr-keyword}{RGB}{0,0,255}
\definecolor{clr-type}{RGB}{43,145,175}
\definecolor{clr-variable}{RGB}{0,0,0}
\definecolor{clr-constant}{RGB}{111,0,138} % macro color
\definecolor{clr-comment}{RGB}{0,128,0}

\lstdefinestyle{mystyle}{
	language=C++,
	backgroundcolor=\color{clr-background},
	basicstyle=\color{clr-text},
	stringstyle=\color{clr-string},
	identifierstyle=\color{clr-variable},
	commentstyle=\color{clr-comment},
	directivestyle=\color{clr-preprocessor},
	keywordstyle=\color{clr-type},
	keywordstyle={[2]\color{clr-constant}},
	tabsize=4,
	showstringspaces=false,
}
\lstset{style=mystyle}
\lstset{directives={\#include}}


\begin{document}
\begin{frame}
\titlepage
\end{frame}

\section*{About this Document}
\begin{frame}\frametitle{About this Document}
\begin{itemize}
\item This document is intended to serve as a useful reference and study tool for the CPE 212 final exam.
\item The code snippets presented in this document are designed to be basic representations of different concepts, and may differ from the methods presented in lectures.
\item Any suggestions for additions or changes can be made to Joshua Bays by emailing \href{mailto:jb0401@uah.edu}{jb0401@uah.edu} or by sending a message via CanvasW
\item This document and all code are under the GPLv3 license, copying, sharing, modifying copies, and sharing modified copies are permitted.
\end{itemize}
\end{frame}

\begin{frame}\frametitle{Table of Contents}
\begin{multicols}{2}
\tableofcontents
\end{multicols}
\end{frame}

\section{Classes}
\begin{frame}\frametitle{Classes - Overview}
\begin{itemize}
\item Class: Custom model of abstract data type (ADT)
\item Object: Instance of a class
\item Member types
	\begin{itemize}
	\item Public: Can be accessed directly from outside of the class
	\item Private: Can be accessed directly only from within the class
	\item Protected: Can be inherited within derived classes
	\end{itemize}
\item Member functions
	\begin{itemize}
	\item Constructor: Initialize an object
	\item Transformers: Change an object's state
	\item Observers: Get (but not change) an object's state
	\item Iterators: Process all components within an ADT
	\item Destructor: Properly clean up and object (Ex: de-allocate memory)
	\end{itemize}
\item Friend function: Function that can access private class members, but is not a member function (used outside of the class)
\end{itemize}
\end{frame}

\begin{frame}\frametitle{Classes - Inheritance}
\begin{itemize}
\item Inheritance: Reuse existing class code for another class
\item Multiple inheritance: Inheriting code from multiple classes
\item Parent/Base class: The class being inherited from
\item Derived class: A class that inherits from another class
\item Virtual function: A member function that can be redefined by an inherited class
\end{itemize}
\end{frame}

\begin{frame}\frametitle{Classes - Code}
Inheritance
\lstinputlisting[basicstyle=\tiny, firstline=3, lastline=20]{code/inheritance.cpp}
\lstinputlisting[basicstyle=\tiny, firstline=30, lastline=36]{code/inheritance.cpp}
\end{frame}

\begin{frame}[fragile]\frametitle{Classes - Code}
Inheritance (cont.)
\begin{center}
\lstinputlisting[basicstyle=\tiny, firstline=22, lastline=28]{code/inheritance.cpp}
\end{center}

Output
\verbatiminput{code/inheritance-output.txt}
\end{frame}

\section{Pointers}
\begin{frame}\frametitle{Pointers - Overview}
\begin{itemize}
\item Pointer: Variable that stores the memory address of another variable
\item Dereferencing: Access the value stored in the location stored in the pointer
\item Static allocation: Memory allocated at compile time
\item Dynamic allocation: Memory allocated during program runtime
\item Heap: Free memory for dynamic allocation
\end{itemize}
\end{frame}

\begin{frame}\frametitle{Pointers - Overview}
\begin{itemize}
\item Memory leak: Memory dynamically allocated but not deallocated
\item Garbage: Locations in memory that can not be accessed any more
\item Inaccessible object: Dynamically allocated variable without a pointer
\item Dangling pointer: A pointer that points to memory that has been deallocated
\end{itemize}
\end{frame}

\begin{frame}\frametitle{Pointers - Dynamic Allocation}
\begin{itemize}
\item Use the new keyword to allocate memory in C++
\item Use the delete keyword on a pointer to deallocate memory in C++
\end{itemize}
\end{frame}

\begin{frame}\frametitle{Pointers - Code}
Static allocation
\lstinputlisting[basicstyle=\tiny, firstline=3]{code/pointer1.cpp}

Output
\verbatiminput{code/pointer1-output.txt}
\end{frame}

\begin{frame}\frametitle{Pointers - Code}
Dynamic Allocation
\lstinputlisting[basicstyle=\tiny, firstline=4]{code/pointer2.cpp}

Output
\verbatiminput{code/pointer2-output.txt}
\end{frame}

\section{Exception Handling}
\begin{frame}\frametitle{Exception Handling - Overview}
\begin{itemize}
\item Robustness: How well a program can recover from an error
\item Error types
	\begin{itemize}
	\item Unexpected user input
	\item Hardware issues
	\item Software issues
	\end{itemize}
\item Ways to handle errors
	\begin{itemize}
	\item Print an error message
	\item Return an unusual value (Ex: -1)
	\item Use a status variable as an error flag
	\item Use assertions to prevent further code execution
	\item Exception handling
	\end{itemize}
\end{itemize}
\end{frame}

\begin{frame}\frametitle{Exception Handling - Overview}
\begin{itemize}
\item Exception: Unexpected event that requires special processing
\item Exception handler: Code designed to address a specific exception
\end{itemize}
\end{frame}

\begin{frame}\frametitle{Exception Handling - Try/Throw/Catch}
\begin{itemize}
\item Try: Execute code that may cause and exception within its own block
\item Throw: If an error is detected terminate the program or execute code to address the exception by ``throwing'' an error
\item Catch: Address the exception based on the type of error provided by the throw statement
\end{itemize}
\end{frame}

\begin{frame}\frametitle{Exception Handling - Code}
Basic exception handling
\lstinputlisting[basicstyle=\tiny, firstline=3]{code/try-throw-catch.cpp}
\end{frame}


\section{Software Engineering}
\begin{frame}\frametitle{Software Engineering - Overview}
\begin{itemize}
\item Attributes of good software
	\begin{itemize}
	\item Works
	\item Can be easily modified
	\item Is reusable
	\item Is completed within time and budget requirements
	\end{itemize}
\item Software engineering: The proper application of the principles of design, production, and maintenance of software
	\begin{itemize}
	\item Technical challenges
	\item Project management
	\end{itemize}
\item Defects in code
	\begin{itemize}
	\item About 1 error is created for every 10 lines of code
	\item 75\% of a code's cost is in maintenance of that code
	\end{itemize}
\item Software process: The process by which software is developed and maintained
\end{itemize}
\end{frame}

\begin{frame}\frametitle{Software Engineering - Parts of Software Development}
\begin{itemize}
\item Requirements: High-level description of the product
\item Specification: Detailed description containing functional requirements and constraints
\item Design: Architectural (high-level) and detailed (low-level) design of the product
\item Implementation: Converting the design into code
\item Testing/Verification: Finding and fixing errors and demonstrating that the product works correctly
\item Postdelivery/Maintenance: Correct errors found by users and enhancing functionality
\end{itemize}
\end{frame}

\begin{frame}\frametitle{Software Engineering - Processes}
\begin{itemize}
\item Waterfall process: Each step of the process is an input for the next step
\item Agile process: Emphasizing individuals/interactions and working software over specific processes in order to enable quick changing and customer collaboration
\item Scrum: Work is designed to be done in short periods called ``sprints,'' with daily work being determined by the needs of the current sprint
\end{itemize}
\end{frame}

\begin{frame}\frametitle{Software Engineering - Testing Overview}
\begin{itemize}
\item Testing: Trying to discover errors within a program
\item Debugging: Removing known errors from a program
\item Driver: A program specifically designed to test a part of code
\item Stub: Dummy code designed to simulate real-life use cases
\item Assertion: A statement that is either true or false
\item Precondition: An assertion that must be true in order for a postcondition to be returned
\item Postcondition: An assertion that is expected from a certain precondition
\end{itemize}
\end{frame}

\begin{frame}\frametitle{Software Engineering - Testing Hierarchy and Types}
\begin{itemize}
\item Deskchecking: Informal checking by the developer
\item Unit testing: Formally testing individual parts of a program by themselves
\item Integration testing: Formally and systematically testing a part of a program within the larger code base
\item Acceptance testing: Testing the program with real data in its real environment
\item Regression testing: Testing a program following modifications
\item Black-box testing: Testing a program by its inputs and outputs
\item Clear-box testing: Testing a program utilizing knowledge of its structure
\end{itemize}
\end{frame}

\begin{frame}\frametitle{Software Engineering - Verification VS Validation}
\begin{itemize}
\item Verification: The program works properly
\item Validation: The program satisfies the needs of the problem
\end{itemize}
\end{frame}

\begin{frame}\frametitle{Software Engineering - Other Notes}
\begin{itemize}
\item When creating interfaces, checking can exist in either in the interface implementation, or within client code (Varies by occasion)
\end{itemize}
\end{frame}

\begin{frame}\frametitle{Software Engineering - Metric-Based Testing}
\begin{itemize}
\item Metric-based testing: Using measurable factors to evaluate how through the testing has been performed
\item Code coverage: How much of the code has been tested
	\begin{itemize}
	\item Necessary, but not sufficient part of software testing
	\item Statement coverage: Percentage of code statements executed
	\item Branch coverage: Does the logical branching execute properly?
	\item Path coverage: How many possible paths can be taken in the code?
	\end{itemize}
\end{itemize}
\end{frame}

\begin{frame}\frametitle{Software Engineering - Tools}
\begin{itemize}
\item gcov: Evaluates code coverage
	\begin{itemize}
	\item Command: g++ -fprofile-arcs -ftest-coverage [object].o -o [executable name]\\
	$[$execute the program$]$\\
	gcov [source file]
	\end{itemize}
\item gdb: debugger
	\begin{itemize}
	\item Command: g++ [source file].cpp -g -o [executable name]\\
	gdb ./[executable]
	\end{itemize}
\item valgrind: bug checker
	\begin{itemize}
	\item Command: valgrind --leak-check=[summary/full] ./[executable]
	\end{itemize}
\end{itemize}
\end{frame}

\section{Stacks}
\begin{frame}\frametitle{Stacks - Overview}
\begin{itemize}
\item Stack: Specially organized list
	\begin{itemize}
	\item LIFO structure (Last In, First Out)
	\item Data entry is only through the top of the stack
	\end{itemize}
\item Basic stack operations:
\begin{itemize}
\item Push: Add an item from the top of the stack
\item Pop: Remove the top item from the stack
\item Top: Observe the top item from the stack
\item IsEmpty: Returns if the stack has no elements on it
\item IsFull: Returns if the stack is at its maximum capacity
\item MakeEmpty: Remove all elements from the stack
\end{itemize}
\item Different methods exist to implement stacks
	\begin{itemize}
	\item Array-based: Less memory used, but harder to resize
	\item Linked node-based: Easier to resize, but more memory used
	\end{itemize}
\end{itemize}
\end{frame}

\begin{frame}\frametitle{Stacks - Code}
Basic stack code (Linked list)
\lstinputlisting[basicstyle=\tiny, firstline=4, lastline=30]{code/stack.cpp}
%\lstinputlisting[basicstyle=\tiny, firstline=12, lastline=27]{code/stack.cpp}
\end{frame}

\begin{frame}\frametitle{Stacks - Code}
Basic stack code (cont.)
\lstinputlisting[basicstyle=\tiny, firstline=47, lastline=55]{code/stack.cpp}
\end{frame}

\begin{frame}\frametitle{Stacks - Code}
Basic stack code (cont.)
\lstinputlisting[basicstyle=\tiny, firstline=57, lastline=75]{code/stack.cpp}
\end{frame}

\begin{frame}\frametitle{Stacks - Code}
Basic stack code (cont.)
\lstinputlisting[basicstyle=\tiny, firstline=77, lastline=99]{code/stack.cpp}
\end{frame}

\begin{frame}\frametitle{Stacks - Code}
Basic stack code (cont.)
\lstinputlisting[basicstyle=\tiny, firstline=101]{code/stack.cpp}
\end{frame}



\section{Lists}
\begin{frame}\frametitle{Lists - Overview}
\begin{itemize}
\item List: Linear collection of homogeneous items
	\begin{itemize}
	\item Can be sorted or unsorted
	\end{itemize}
\item Basic list operations:
	\begin{itemize}
	\item IsEmpty: Returns if the list has no elements in it
	\item IsFull: Returns if the list is at its maximum capacity
	\item Length: Returns the amount of elements in the list
	\item Insert: Add an item to the list (May implement sorting)
	\item Delete: Delete an item from the list
	\item IsPresent: Check if an item exists in the list
	\end{itemize}
\end{itemize}
\end{frame}

\begin{frame}\end{frame}
\begin{frame}\end{frame}

\section{Queues}

\section{Reusable Code}
\begin{frame}\frametitle{Reusable Code - Generic Programming}
\begin{itemize}
\item Generic programming: Allowing multiple types to be used as parameters by using a template
	\begin{itemize}
	\item Template: Code that gets expanded at compile time with the item types implemented within that code
	\end{itemize}
\item Uses
	\begin{itemize}
	\item Code can be reused without function overloading
	\item Code can be reused for multiple cases
	\end{itemize}
\end{itemize}
\end{frame}

\begin{frame}\frametitle{Reusable Code - Code}
Basic template code (Function template)
\lstinputlisting[basicstyle=\tiny, firstline=3]{code/generic-programming-1.cpp}

Output
\verbatiminput{code/generic-programming-1-output.txt}
\end{frame}

\begin{frame}\frametitle{Reusable Code - Code}
Basic template code (Function template)
\lstinputlisting[basicstyle=\tiny, firstline=3]{code/generic-programming-2.cpp}

Output
\verbatiminput{code/generic-programming-2-output.txt}
\end{frame}

\begin{frame}\frametitle{Reusable Code - Overloading}
\begin{itemize}
\item Function overloading: Using the same function name for different parameter types (not return types)
\item Operator overloading: Extending an operator's functionality to work with custom data types and objects
	\begin{itemize}
	\item The operators . .* :: ?: cannot be overloaded
	\end{itemize}
\end{itemize}
\end{frame}

\begin{frame}\frametitle{Reusable Code - Code}
Basic function overloading code
\lstinputlisting[basicstyle=\tiny, firstline=3]{code/function-overloading.cpp}

Output
\verbatiminput{code/function-overloading-output.txt}
\end{frame}

\begin{frame}\frametitle{Reusable Code - Code}
Basic operator overloading code
\lstinputlisting[basicstyle=\tiny, firstline=3]{code/operator-overloading.cpp}

Output
\verbatiminput{code/operator-overloading-output.txt}
\end{frame}

\section{Recursion}
\begin{frame}\frametitle{Recursion - Overview}
\begin{itemize}
\item Recursion: A function that implements itself somewhere in execution
	\begin{itemize}
	\item Direct recursion: A function directly calls itself
	\item Indirect recursion: A series of functions is implemented in which the originating function is called at some point
	\end{itemize}
\item Base case: A nonrecursive instance of a recurring function
\item Recursive case: A case of a recurring function that can be expressed in terms of itself
\item Tail recursion: No statements are executed after the return from a recursive call
\end{itemize}
\end{frame}

\begin{frame}\frametitle{Recursion - Implementation}
\begin{itemize}
\item Steps
	\begin{itemize}
	\item Understand the problem
	\item Determine the size of the problem
	\item Solve the base case
	\item Solve the general case using smaller versions of the general case
	\end{itemize}
\item Use cases
	\begin{itemize}
	\item Shallow depth of recursion (high cost to perform recursion)
	\item The amount of recursive cases grow slowly
	\item The recursive solution is simpler or shorter than the nonrecursive
	\end{itemize}
\item Limitations
	\begin{itemize}
	\item Infinite recursion (Can cause stack overflow)
	\item Sometimes an iterative method makes more sense to implement
	\end{itemize}
\end{itemize}
\end{frame}

\begin{frame}\frametitle{Recursion - Code}
Basic recursive code (Fibonacci sequence)
\lstinputlisting[basicstyle=\tiny, firstline=3]{code/recursion.cpp}
Output
\verbatiminput{code/recursion-output.txt}
\end{frame}


\section{Trees}
\begin{frame}\frametitle{Trees - Overview}
\begin{itemize}
\item 
\end{itemize}
\end{frame}

\section{Heaps}
\begin{frame}\frametitle{Heaps - Overview}
\begin{itemize}
\item Heap: A complete binary tree that has either the greatest value (max heap) or the least greatest value (min heap) as the root node
\item Basic heap operations
	\begin{itemize}
	\item Heapify: Rearrange the heap to maintain its order of max heap or min heap
	\item Insert: Add an item to the heap and possibly heapify
	\item Delete: Remove the root node, make the last node the root, and heapify
	\end{itemize}
\item Heaps are typically implemented as arrays
	\begin{itemize}
	\item root: array[0]
	\item parent of ith node: array[(i-1)/2]
	\item left child of ith node: array[(i*2)+1]
	\item right child of ith node: array[(i*2)+2]
	\end{itemize}
\item Uses
	\begin{itemize}
	\item Priority queues
	\item Sorting algorithms (heap sort)
	\end{itemize}
\end{itemize}
\end{frame}

\begin{frame}\frametitle{Heaps - Code}
Basic heap sort code
\lstinputlisting[basicstyle=\tiny, firstline=6]{code/heap.cpp}
\end{frame}

\begin{frame}\frametitle{Sorting - Code}
Output
\verbatiminput{code/heap-output.txt}
\end{frame}

\begin{frame}\frametitle{Heaps - Complexity Table}
\begin{center}
\begin{tabular}{|c|c|c|c|c|}
\hline
Data structure & Insertion & Deletion & Search & Indexing\\\hline
BST (Average case) & $O(\log n)$ & $O(\log n)$ & $O(\log n)$ & \\\hline
BST (Worst case) & $O(n)$ & $O(n)$ & $O(n)$ & \\\hline
Heap & $O(\log n)$ & $O(\log n)$ & $O(n)$ & $O(1)$ \\\hline
Array & $O(n)$ & $O(n)$ & $O(n)$ & $O(1)$ \\\hline
Linked List & $O(1)$ (at head) & $O(1)$ (at head) & $O(n)$ & $O(n)$ \\\hline
\end{tabular}
\end{center}
\end{frame}

\section{Graphs}
\begin{frame}\frametitle{Graphs - Overview}
\begin{itemize}
\item Graph: Data structure of vertices/nodes and edges connecting the nodes
	\begin{itemize}
	\item Formally represented as $G = (V, E)$, where $V$ is a set of vertices, and $E$ is a set of edges
	\end{itemize}
\item Vertex/Node: Point on the graph
\item Edge: A pair of vertices that represents a connection between them
	\begin{itemize}
	\item Formally represented as $\{V_1, V_2\}$, where $V_1$ and $V_2$ are 2 vertices
	\end{itemize}
\item Degree: Number of edges touching a vertex
\item Path: Series of edges that can be traversed in order to travel between 2 vertices
\item Neighbor/Adjacent: Vertex directly connected to another vertex by an edge
\item Assuming there are no self-connected edges, if $|V| = n$, then for a directed $0 \leq |E| \leq n(n - 1)$, and for an undirected graph $0 \leq |E| \leq \frac{n(n - 1)}{2}$
\end{itemize}
\end{frame}

\begin{frame}\frametitle{Graphs - Types}
\begin{itemize}
\item Undirected graph: Edges can be traversed in both directions ($\{A, B\} = \{B, A\}$ if $A \neq B$)
\item Directed graph: Edges cannot be traversed in both directions ($\{A, B\} \neq \{B, A\}$ if $A \neq B$)
\item Connected: Every vertex has at least 1 path with another vertex
\item Strongly connected: Every vertex has an edge that connects to every other vertex
\item Weighted graph: Every edge has an associated numerical value
\item Unweighted graph: Every edge's associate numerical value is equal or no value is associated with an edge
\end{itemize}
\end{frame}

\begin{frame}\frametitle{Graphs - Storage and Representation}
\begin{itemize}
\item Edge list: Create a set of vertex objects and a list with items containing a pointer to a source node and destination node
	\begin{itemize}
	\item Efficiency of finding all adjacent nodes: $O(|E|)$
	\item Efficiency of finding if nodes are connected: $O(|E|)$
	\end{itemize}
\item Adjacency matrix: $|V| \times |V|$ matrix where $A_{i j} = \begin{cases}1 & \text{i and j are connected}\\ 0 & \text{otherwise}\end{cases}$
	\begin{itemize}
	\item Efficiency of finding all adjacent nodes: $O(|V|)$
	\item Efficiency of finding if nodes are connected: $O(|V|)$
	\item Efficiency of finding if nodes are connected (hash table): $O(|1|)$
	\item Space efficiency: $O(|V|^2)$
	\end{itemize}
\end{itemize}
\end{frame}

\begin{frame}\frametitle{Graphs - Storage and Representation}
\begin{itemize}
\item Adjacency list: Dynamically allocated list of edges for each vertex
	\begin{itemize}
	\item Less space used when compared to adjacency matrix
	\item Efficiency of finding all adjacent nodes: $O(|V|)$
	\item Efficiency of finding if nodes are connected: $O(|V|)$
	\item Efficiency of finding if nodes are connected (binary search tree): $O(|\log V|)$
	\item Space efficiency: $O(|E|)$
	\end{itemize}
\end{itemize}
\end{frame}

\begin{frame}\frametitle{Graphs - Searching}
\begin{itemize}
\item Depth-First Search (DFS): Traverse a branch to the deepest point before going back
\item Breadth-First Seach (BFS): Look at all paths of the same depth before going to the next level
\item Greedy algorithm: Choose the most locally optimal choice each step of an algorithm in order to find a generally globally efficient result
	\begin{itemize}
	\item Usually simpler to program
	\item Usually not the most efficient overall method
	\end{itemize}
\end{itemize}
\end{frame}

\section{Searching}
\begin{frame}\frametitle{Searching - Overview}
\begin{itemize}
\item Searching: Finding a specific item from a set of data
	\begin{itemize}
	\item Efficiency: Program performance is improved
	\item Data retrieval: Specific data is quickly found in a large dataset
	\item Problem solving: Data needs to be found in order to solve problems
	\end{itemize}
\item Different methods of searching
	\begin{itemize}
	\item Linear search
	\item Binary search
	\item Hashing
	\end{itemize}
\end{itemize}
\end{frame}

\begin{frame}\frametitle{Searching - Linear Search}
\begin{itemize}
\item Linear search: Sequentially search through a set of data until the value is found
\item Complexity:
	\begin{itemize}
	\item Best case: $O(1)$ (The first element)
	\item Worst case: $O(n)$ (The last element)
	\end{itemize}
\item Use cases:
	\begin{itemize}
	\item Small dataset
	\item Unordered datasets
	\item Linked lists
	\end{itemize}
\end{itemize}
\end{frame}

\begin{frame}\frametitle{Searching - Code}
Basic linear search code
\lstinputlisting[basicstyle=\tiny, firstline=3]{code/linear-search.cpp}

Output
\verbatiminput{code/linear-search-output.txt}
\end{frame}

\begin{frame}\frametitle{Searching - Binary Search}
\begin{itemize}
\item Binary search: Continually divide the search area in half, comparing the middle value to the target value
\item Complexity
	\begin{itemize}
	\item Best case: $O(1)$ (The first element)
	\item Worst case: $O(\log n)$ (The last element)
	\end{itemize}
\item Use cases:
	\begin{itemize}
	\item Data must be sorted
	\item Random access should be a constant time function
	\end{itemize}
\end{itemize}
\end{frame}

\begin{frame}\frametitle{Searching - Code}
Basic binary search code
\lstinputlisting[basicstyle=\tiny, firstline=3]{code/binary-search.cpp}

Output
\verbatiminput{code/binary-search-output.txt}
\end{frame}

\begin{frame}\frametitle{Searching - Hashing}
\begin{itemize}
\item Hashing: Data storage technique designed to allow $O(1)$ search time
	\begin{itemize}
	\item Assign key-value pairs through a function to data inputs
	\item Hash function used to store the element and to find if the element exists in the dataset
	\item Tradeoff of memory space for access speed
	\end{itemize}
\item Collision: Repeated outputs for different inputs
	\begin{itemize}
	\item Must be addressed with hash function (Ex: Offsetting the value)
	\end{itemize}
\end{itemize}
\end{frame}

\begin{frame}\frametitle{Searching - Code}
Basic hashing code
\lstinputlisting[basicstyle=\tiny, firstline=3, lastline=25]{code/hashing.cpp}
\end{frame}

\begin{frame}\frametitle{Searching - Code}
Basic hashing code (Cont.)
\lstinputlisting[basicstyle=\tiny, firstline=27]{code/hashing.cpp}
Output
\verbatiminput{code/hashing-output.txt}
\end{frame}

\section{Sorting}
\begin{frame}\frametitle{Sorting - Overview}
\begin{itemize}
\item Sorting: Organizing data based on its value
	\begin{itemize}
	\item Usually based on numeric value or alphabetical value
	\end{itemize}
\item Efficiency
	\begin{itemize}
	\item Speed: How many comparisons are made and how many swaps are required
	\item Space: How much memory is required
	\item More memory in usually traded for faster speed
	\end{itemize}
\item Divide and conquer: Method some algorithms use to sort smaller sections of data and merging them back together
\end{itemize}
\end{frame}

\begin{frame}\frametitle{Sorting - Selection Sort}
\begin{itemize}
\item Selection sort: Continually swap the smallest/largest unsorted value with the first unsorted element
	\begin{itemize}
	\item Completes redundant swaps
	\end{itemize}
\item Efficiency
	\begin{itemize}
	\item Best case: $O(n^2)$
	\item Worst case: $O(n^2)$
	\item Average case: $O(n^2)$
	\item $n - 1$ swaps will always be performed
	\end{itemize}
\end{itemize}
\end{frame}

\begin{frame}\frametitle{Sorting - Code}
Basic selection sort code
\lstinputlisting[basicstyle=\tiny, firstline=6]{code/selection-sort.cpp}
\end{frame}

\begin{frame}\frametitle{Sorting - Code}
Output
\verbatiminput{code/selection-sort-output.txt}
\end{frame}

\begin{frame}\frametitle{Sorting - Bubble Sort}
\begin{itemize}
\item Bubble sort: Move the smallest/largest value to the front/end of the list
	\begin{itemize}
	\item Compare each item to its immediate successor
	\item The next smallest/largest value will be moved to its correct place each pass through
	\end{itemize}
\item Efficiency
	\begin{itemize}
	\item Best case: $O(n)$, $0$ swaps
	\item Worst case: $O(n^2)$, $\frac{n^2}{2}$ swaps
	\item Average case: $O(n^2)$, $(\frac{1}{2})(\frac{n^2}{2})$ swaps
	\end{itemize}
\end{itemize}
\end{frame}

\begin{frame}\frametitle{Sorting - Code}
Basic bubble sort code
\lstinputlisting[basicstyle=\tiny, firstline=6]{code/bubble-sort.cpp}
\end{frame}

\begin{frame}\frametitle{Sorting - Code}
Output
\verbatiminput{code/bubble-sort-output.txt}
\end{frame}

\begin{frame}\frametitle{Sorting - Insertion Sort}
\begin{itemize}
\item Insertion sort: Move each item to its proper place in reference to its predecessors
	\begin{itemize}
	\item Assumes the first item is sorted
	\end{itemize}
\item Efficiency
	\begin{itemize}
	\item Best case: $O(n^2)$
	\item Worst case: $O(n^2)$, $\frac{n^2}{2}$ swaps
	\item Average case: $O(n^2)$, $(\frac{1}{2})(\frac{n^2}{2})$ swaps
	\end{itemize}
\end{itemize}
\end{frame}

\begin{frame}\frametitle{Sorting - Code}
Basic insertion sort code
\lstinputlisting[basicstyle=\tiny, firstline=6]{code/insertion-sort.cpp}
\end{frame}

\begin{frame}\frametitle{Sorting - Code}
Output
\verbatiminput{code/insertion-sort-output.txt}
\end{frame}

\begin{frame}\frametitle{Sorting - Quick Sort}
\begin{itemize}
\item Quick sort: Divide and conquer sorting algorithm by partitioning around a pivot and recursively sorting each pivot
	\begin{itemize}
	\item Values less than the pivot go to one side, and values greater than the pivot go to the other side
	\item Base case: A partition has one element
	\end{itemize}
\item Efficiency
	\begin{itemize}
	\item Best case: $O(n\log n)$ (Each split generates equally-sized partitions)
	\item Worst case: $O(n^2)$ (Mostly sorted)
	\end{itemize}
\end{itemize}
\end{frame}

\begin{frame}\frametitle{Sorting - Code}
Basic quick sort code
\lstinputlisting[basicstyle=\tiny, firstline=6]{code/quick-sort.cpp}
\end{frame}

\begin{frame}\frametitle{Sorting - Code}
Output
\verbatiminput{code/quick-sort-output.txt}
\end{frame}

\begin{frame}\frametitle{Sorting - Merge Sort}
\begin{itemize}
\item Merge sort: Continually divide the dataset into smaller datasets and sorting and merging them back together
\item Efficiency
	\begin{itemize}
	\item Best case: $O(n \log n)$
	\item Worst case: $O(n \log n)$
	\end{itemize}
\end{itemize}
\end{frame}

\begin{frame}\frametitle{Sorting - Code}
Basic merge sort code
\lstinputlisting[basicstyle=\tiny, firstline=7]{code/merge-sort.cpp}
\end{frame}

\begin{frame}\frametitle{Sorting - Code}
Output
\verbatiminput{code/merge-sort-output.txt}
\end{frame}

\begin{frame}\frametitle{Radix Sort}
\begin{itemize}
\item Sorting elements in a dataset based on its value within a known range (Ex: Leading digit)
\item Efficiency
	\begin{itemize}
	\item $O(kn)$, where $k$ is the amount of times each data set is sorted
	\end{itemize}
\end{itemize}
\end{frame}

\begin{frame}\frametitle{Sorting - Code}
Basic radix sort code
\lstinputlisting[basicstyle=\tiny, firstline=6]{code/radix-sort.cpp}
\end{frame}

\begin{frame}\frametitle{Sorting - Code}
Output
\verbatiminput{code/radix-sort-output.txt}
\end{frame}

\begin{frame}\frametitle{Heap Sort}
\begin{itemize}
\item Heap sort: Get and remove the maximum value from a sorted heap, then heapify the updated heap
\item Efficency
	\begin{itemize}
	\item Heap construction: $O(n)$
	\item Heapify once: $O(\log n)$
	\item Complete sorting: $O(n \log n)$
	\item Intiial ordering does not affect efficency
	\end{itemize}
\end{itemize}
\end{frame}

%\begin{frame}\frametitle{Sorting - Code}
%Basic heap sort code
%\lstinputlisting[basicstyle=\tiny, firstline=5]{code/heap-sort.cpp}
%\end{frame}

%\begin{frame}\frametitle{Sorting - Code}
%Output
%\verbatiminput{code/heap-sort-output.txt}
%\end{frame}

\begin{frame}\frametitle{Sorting - Efficiency Table}
Efficiencies of sorting algorithms
\begin{center}
\begin{tabular}{|c|c|c|c|}
\hline Algorithm & Best & Average & Worst\\\hline
Selection & $O(n^2)$ & $O(n^2)$ & $O(n^2)$\\\hline
Bubble & $O(n)$ & $O(n^2)$ & $O(n^2)$\\\hline
Insertion & $O(n^2)$ & $O(n^2)$ & $O(n^2)$\\\hline
Quick & $O(n\log n)$ & $O(n\log n)$ & $O(n^2)$\\\hline
Radix & $O(nk)$ & $O(nk)$ & $O(nk)$\\\hline
Heap & $O(n\log n)$ & $O(n\log n)$ & $O(n\log n)$\\\hline
\end{tabular}
\end{center}
\end{frame}


\section{STL}
\begin{frame}\frametitle{STL - Overview}
\begin{itemize}
\item Standard Template Library: Standardized pre-written templates
	\begin{itemize}
	\item Already tested and debugged
	\item Efficient performance
	\end{itemize}
\item Sequence containers: Elements have a set order (indexed)
	\begin{itemize}
	\item Can be contiguous in memory (array, vector) or not contiguous (deque, list)
	\end{itemize}
\item Associative containers: Elements are based on keys
	\begin{itemize}
	\item Usually implemented with a balanced binary tree
	\item Can limit keys to one instance (sets, maps) or allow duplicate keys (multisets, mulitmaps)
	\end{itemize}
\end{itemize}
\end{frame}

\begin{frame}\frametitle{STL - Vectors}
\begin{itemize}
\item Vector: Dynamically sized array of a specified data type
\item Basic vector operations:
	\begin{itemize}
	\item vector$<$T$>$: Constructor
	\item $\sim$vector$<$T$>$: Destructor
	\item size(): Return the amount of items in the vector
	\item empty(): Return if there are no items in the vector
	\item push\_back(T item): Append an item to the end of the vector
	\item pop\_back(): Remove the end item of the vector
	\item clear(): Remove all elements from the vector
	\item begin(): Iterator of the front element
	\item end(): Iterator of the end element
	\end{itemize}
\end{itemize}
\end{frame}

\begin{frame}\frametitle{STL - Sets}
\begin{itemize}
\item Set: Sorted storage of key values
	\begin{itemize}
	\item Logarithmic search performance
	\item Direct changing of elements is not permitted (remove the old value and add the new value)
	\end{itemize}
\item Multiset: Set that allows duplicate values
\end{itemize}
\end{frame}

\begin{frame}\frametitle{STL - Sets}
\begin{itemize}
\item Basic set/multiset operations:
	\begin{itemize}
	\item set$<$T$>$ / multiset$<$T$>$: Constructor
	\item $\sim$set$<$T$>$ / $\sim$multiset$<$T$>$: Destructor
	\item size(): Return the amount of items in the set
	\item empty(): Return if there are no items in the set
	\item find(T item): Return (first) position of the provided item
	\item count(T item): Return how many items of the passed value exist in the set
	\item insert(T item): Add an item of the passed value and return the index it was inserted to
	\item erase(T item): Remove all items of the passed value from the set/multiset and return how many items were removed
	\item clear(): Remove all items from the set
	\end{itemize}
\end{itemize}
\end{frame}

\begin{frame}\frametitle{STL - Maps}
\begin{itemize}
\item Map: Sorted storage of key-value pairs
\item Multimap: Map that allows duplicate keys
\end{itemize}
\end{frame}

\begin{frame}\frametitle{STL - Maps}
\begin{itemize}
\item Basic map/multimap operations:
	\begin{itemize}
	\item map$<$T, T$>$ / multimap$<$T, T$>$: Constructor (Provide both key and value data types)
	\item map$<$T Op$>$ / multimap$<$T Op$>$:\\Constructor with sorting operation (Ex: map$<$int greater$<$int$>>$ is a descending map)
	\item $\sim$map$<$T, T$>$ / $\sim$multimap$<$T, T$>$: Destructor
	\item size(): Return the amount of items in the map
	\item empty(): Return if there are no items in the map
	\item find(T item): Return (first) position of the provided item
	\item count(T item): Return how many items of the passed value exist in the map
	\item insert(T item): Add an item of the passed value and return the index it was inserted to
	\item erase(T item): Remove all items of the passed value from the set/multiset and return how many items were removed
	\item clear(): Remove all items from the map
	\item map[item]: Return the value associated with the key (map only)
	\end{itemize}
\end{itemize}
\end{frame}


\end{document}
