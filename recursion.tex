\section{Recursion}
\begin{frame}\frametitle{Recursion - Overview}
\begin{itemize}
\item Recursion: A function that implements itself somewhere in execution
	\begin{itemize}
	\item Direct recursion: A function directly calls itself
	\item Indirect recursion: A series of functions is implemented in which the originating function is called at some point
	\end{itemize}
\item Base case: A nonrecursive instance of a recurring function
\item Recursive case: A case of a recurring function that can be expressed in terms of itself
\item Tail recursion: No statements are executed after the return from a recursive call
\end{itemize}
\end{frame}

\begin{frame}\frametitle{Recursion - Implementation}
\begin{itemize}
\item Steps
	\begin{itemize}
	\item Understand the problem
	\item Determine the size of the problem
	\item Solve the base case
	\item Solve the general case using smaller versions of the general case
	\end{itemize}
\item Use cases
	\begin{itemize}
	\item Shallow depth of recursion (high cost to perform recursion)
	\item The amount of recursive cases grow slowly
	\item The recursive solution is simpler or shorter than the nonrecursive
	\end{itemize}
\item Limitations
	\begin{itemize}
	\item Infinite recursion (Can cause stack overflow)
	\item Sometimes an iterative method makes more sense to implement
	\end{itemize}
\end{itemize}
\end{frame}

\begin{frame}\frametitle{Recursion - Code}
Basic recursive code (Fibonacci sequence)
\lstinputlisting[basicstyle=\tiny, firstline=3]{code/recursion.cpp}
Output
\verbatiminput{code/recursion-output.txt}
\end{frame}
