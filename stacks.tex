\section{Stacks}
\begin{frame}\frametitle{Stacks - Overview}
\begin{itemize}
\item Stack: Specially organized list
	\begin{itemize}
	\item LIFO structure (Last In, First Out)
	\item Data entry is only through the top of the stack
	\end{itemize}
\item Basic stack operations:
	\begin{itemize}
	\item Push: Add an item from the top of the stack
	\item Pop: Remove the top item from the stack
	\item Top: Observe the top item from the stack
	\item IsEmpty: Returns if the stack has no elements on it
	\item IsFull: Returns if the stack is at its maximum capacity
	\item MakeEmpty: Remove all elements from the stack
\end{itemize}
\item Different methods exist to implement stacks
	\begin{itemize}
	\item Array-based: Less memory used, but harder to resize
	\item Linked node-based: Easier to resize, but more memory used
	\end{itemize}
\end{itemize}
\end{frame}

\begin{frame}\frametitle{Stacks - Code}
Basic stack code (Linked list)
\lstinputlisting[basicstyle=\tiny, firstline=4, lastline=30]{code/stack.cpp}
%\lstinputlisting[basicstyle=\tiny, firstline=12, lastline=27]{code/stack.cpp}
\end{frame}

\begin{frame}\frametitle{Stacks - Code}
Basic stack code (cont.)
\lstinputlisting[basicstyle=\tiny, firstline=47, lastline=55]{code/stack.cpp}
\end{frame}

\begin{frame}\frametitle{Stacks - Code}
Basic stack code (cont.)
\lstinputlisting[basicstyle=\tiny, firstline=57, lastline=75]{code/stack.cpp}
\end{frame}

\begin{frame}\frametitle{Stacks - Code}
Basic stack code (cont.)
\lstinputlisting[basicstyle=\tiny, firstline=77, lastline=99]{code/stack.cpp}
\end{frame}

\begin{frame}\frametitle{Stacks - Code}
Basic stack code (cont.)
\lstinputlisting[basicstyle=\tiny, firstline=101]{code/stack.cpp}
\end{frame}
